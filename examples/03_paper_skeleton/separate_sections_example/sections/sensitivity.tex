% ============================================================================
% Sensitivity Analysis Section - Structure Level Template
% MCM-Analysis Skill v1.3.0
% ============================================================================

\section{Sensitivity Analysis}

% ============================================================================
% 【本节目标】1-1.5页,验证模型鲁棒性
% 【关键内容】
%   - 参数敏感性测试 (±5%, ±10%)
%   - 模型验证
%   - 鲁棒性结论
% ============================================================================

\subsection{Parameter Sensitivity}

% TODO: 选择2-4个关键参数进行敏感性分析

% --- 敏感性分析结果图 ---
% \begin{figure}[H]
%     \centering
%     % \includegraphics[width=0.8\textwidth]{figures/sensitivity.pdf}
%     \caption{Sensitivity analysis of key parameters}
%     \label{fig:sensitivity}
% \end{figure}

% TODO: 敏感性分析结果表
% \begin{table}[H]
% \centering
% \begin{tabular}{lccc}
% \toprule
% \textbf{Parameter} & \textbf{-10\%} & \textbf{Baseline} & \textbf{+10\%} \\
% \midrule
% Parameter 1 & [result] & [result] & [result] \\
% Parameter 2 & [result] & [result] & [result] \\
% \bottomrule
% \end{tabular}
% \caption{Impact of parameter perturbation on model outputs}
% \label{tab:sensitivity}
% \end{table}

% TODO: 解释哪些参数影响大,哪些影响小

\subsection{Model Validation}

% TODO: 模型验证方法
% 选项:
% - 与历史数据对比
% - 交叉验证
% - 极端情况测试
% - 合理性检查

% --- 验证结果 ---
% \begin{figure}[H]
%     \centering
%     % \includegraphics[width=0.7\textwidth]{figures/validation.pdf}
%     \caption{Comparison between model predictions and actual data}
%     \label{fig:validation}
% \end{figure}

\subsection{Robustness Conclusion}

% TODO: 总结模型鲁棒性
% 示例: "Our sensitivity analysis demonstrates that the model is robust 
% to reasonable variations in key parameters. The output changes by less 
% than X% when parameters vary within ±10%."
