% ============================================================================
% Strengths and Weaknesses Section - Structure Level Template
% MCM-Analysis Skill v1.3.0
% ============================================================================

\section{Strengths and Weaknesses}

% ============================================================================
% 【本节目标】0.5-1页,客观评估模型
% 【平衡原则】优点和缺点数量大致相当,缺点要提供改进方向
% ============================================================================

\subsection{Strengths}

\begin{itemize}
    % TODO: 列出3-5个优点,每个都要有具体解释
    
    \item \textbf{[Strength 1 Title]:}
    % TODO: 具体说明为什么这是优点
    
    \item \textbf{[Strength 2 Title]:}
    % TODO: 具体说明为什么这是优点
    
    \item \textbf{[Strength 3 Title]:}
    % TODO: 具体说明为什么这是优点
    
    % 常见优点类型:
    % - Comprehensive consideration of multiple factors
    % - Supported by real-world data
    % - Robust to parameter variations
    % - Practical applicability
    % - Novel approach / methodology innovation
    % - Validated against known benchmarks
    
\end{itemize}

\subsection{Weaknesses}

\begin{itemize}
    % TODO: 列出2-4个缺点,每个都要提供可能的改进方向
    
    \item \textbf{[Weakness 1 Title]:}
    % TODO: 描述局限性
    
    \textit{Potential Improvement:} % TODO: 如何改进
    
    \item \textbf{[Weakness 2 Title]:}
    % TODO: 描述局限性
    
    \textit{Potential Improvement:} % TODO: 如何改进
    
    \item \textbf{[Weakness 3 Title]:}
    % TODO: 描述局限性
    
    \textit{Potential Improvement:} % TODO: 如何改进
    
    % 常见缺点类型:
    % - Simplified assumptions may not capture all real-world complexity
    % - Limited by available data quality/quantity
    % - Computational complexity restricts scalability
    % - Requires domain expertise for parameter tuning
    % - Geographic/temporal scope limitations
    
\end{itemize}
