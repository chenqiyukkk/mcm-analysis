% ============================================================================
% Assumptions Section - Structure Level Template
% MCM-Analysis Skill v1.3.0
% ============================================================================

\section{Assumptions and Justifications}

% ============================================================================
% 【本节目标】1-1.5页,包含假设列表和符号表
% 【假设数量】通常5-8个,每个都需要合理性论证
% ============================================================================

We make the following assumptions to simplify the problem while maintaining practical validity:

\begin{enumerate}[label=\textbf{A\arabic*.}]
    
    % --- 假设1: 数据可靠性 ---
    \item \textbf{Data Reliability:}
    % TODO: 关于数据来源和质量的假设
    
    \textit{Justification:} % TODO: 为什么这个假设合理
    
    % --- 假设2: 系统边界 ---
    \item \textbf{System Boundary:}
    % TODO: 关于系统范围和边界条件的假设
    
    \textit{Justification:} % TODO: 为什么这个假设合理
    
    % --- 假设3: 行为假设 ---
    \item \textbf{Behavioral Assumption:}
    % TODO: 关于参与者/系统行为的假设(如理性决策)
    
    \textit{Justification:} % TODO: 为什么这个假设合理
    
    % --- 假设4: 简化假设 ---
    \item \textbf{Simplification:}
    % TODO: 为简化模型而做的假设
    
    \textit{Justification:} % TODO: 为什么这个简化不影响结论
    
    % --- 假设5: 时间/空间假设 ---
    \item \textbf{Temporal/Spatial Scope:}
    % TODO: 关于时间范围或空间范围的假设
    
    \textit{Justification:} % TODO: 为什么这个假设合理

\end{enumerate}

\subsection{Notation}

% TODO: 填充符号表,包含所有重要变量
\begin{table}[H]
\centering
\begin{tabular}{clc}
\toprule
\textbf{Symbol} & \textbf{Description} & \textbf{Unit} \\
\midrule
$x$ & [Description of variable x] & [unit] \\
$y$ & [Description of variable y] & [unit] \\
$\alpha$ & [Description of parameter alpha] & - \\
$\beta$ & [Description of parameter beta] & - \\
% TODO: 添加更多符号
\bottomrule
\end{tabular}
\caption{Notation used in this paper}
\label{tab:notation}
\end{table}

\textit{Note: Additional variables will be defined in context within each section.}
