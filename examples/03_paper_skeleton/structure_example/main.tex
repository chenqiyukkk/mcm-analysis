\documentclass[12pt]{article}

% ============================================================================
% MCM/ICM Paper
% Problem: C
% Year: 2026
% ============================================================================

% ----- Preamble -----
% ============================================================================
% MCM/ICM LaTeX Preamble - Standard Configuration
% Generated by MCM-Analysis Skill v1.3.0
% ============================================================================

% ----- Page Layout -----
\usepackage[letterpaper, margin=1in]{geometry}
\usepackage{fancyhdr}

% ----- Math Packages -----
\usepackage{amsmath, amssymb, amsthm}
\usepackage{mathtools}

% ----- Graphics and Tables -----
\usepackage{graphicx}
\usepackage{float}
\usepackage{booktabs}
\usepackage{tabularx}
\usepackage{multirow}
\usepackage{subcaption}

% ----- Code and Algorithms -----
\usepackage{algorithm}
\usepackage{algpseudocode}
\usepackage{listings}

% ----- References and Links -----
\usepackage[hidelinks]{hyperref}
\usepackage{cite}
\usepackage{url}

% ----- Formatting -----
\usepackage{enumitem}
\usepackage{setspace}
\usepackage{xcolor}
\usepackage{lastpage}

% ----- CJK Support (for Chinese content) -----
\usepackage{xeCJK}
\setCJKmainfont{SimSun}[AutoFakeBold=true, AutoFakeSlant=true]
\setCJKsansfont{SimHei}
\setCJKmonofont{FangSong}

% ----- Header/Footer Setup -----
\pagestyle{fancy}
\fancyhf{}
\lhead{Team \# XXXXXXX}  % Replace XXXXXXX with your team number
\rhead{Page \thepage\ of \pageref{LastPage}}
\renewcommand{\headrulewidth}{0pt}

% ----- Custom Commands -----
\newcommand{\teamnum}{XXXXXXX}  % Replace with your team number
\newtheorem{theorem}{Theorem}
\newtheorem{lemma}{Lemma}
\newtheorem{definition}{Definition}

% ----- Code Listing Style -----
\lstset{
    basicstyle=\ttfamily\small,
    breaklines=true,
    frame=single,
    numbers=left,
    numberstyle=\tiny,
    tabsize=4
}

% ----- Colors for TODO markers -----
\definecolor{todocolor}{RGB}{255, 100, 100}
\newcommand{\TODO}[1]{\textcolor{todocolor}{\textbf{[TODO: #1]}}}


% ============================================================================
\begin{document}

% ----- Preamble Content -----
% ============================================================================
% MCM/ICM LaTeX Preamble - Standard Configuration
% Generated by MCM-Analysis Skill v1.3.0
% ============================================================================

% ----- Page Layout -----
\usepackage[letterpaper, margin=1in]{geometry}
\usepackage{fancyhdr}

% ----- Math Packages -----
\usepackage{amsmath, amssymb, amsthm}
\usepackage{mathtools}

% ----- Graphics and Tables -----
\usepackage{graphicx}
\usepackage{float}
\usepackage{booktabs}
\usepackage{tabularx}
\usepackage{multirow}
\usepackage{subcaption}

% ----- Code and Algorithms -----
\usepackage{algorithm}
\usepackage{algpseudocode}
\usepackage{listings}

% ----- References and Links -----
\usepackage[hidelinks]{hyperref}
\usepackage{cite}
\usepackage{url}

% ----- Formatting -----
\usepackage{enumitem}
\usepackage{setspace}
\usepackage{xcolor}
\usepackage{lastpage}

% ----- CJK Support (for Chinese content) -----
\usepackage{xeCJK}
\setCJKmainfont{SimSun}[AutoFakeBold=true, AutoFakeSlant=true]
\setCJKsansfont{SimHei}
\setCJKmonofont{FangSong}

% ----- Header/Footer Setup -----
\pagestyle{fancy}
\fancyhf{}
\lhead{Team \# XXXXXXX}  % Replace XXXXXXX with your team number
\rhead{Page \thepage\ of \pageref{LastPage}}
\renewcommand{\headrulewidth}{0pt}

% ----- Custom Commands -----
\newcommand{\teamnum}{XXXXXXX}  % Replace with your team number
\newtheorem{theorem}{Theorem}
\newtheorem{lemma}{Lemma}
\newtheorem{definition}{Definition}

% ----- Code Listing Style -----
\lstset{
    basicstyle=\ttfamily\small,
    breaklines=true,
    frame=single,
    numbers=left,
    numberstyle=\tiny,
    tabsize=4
}

% ----- Colors for TODO markers -----
\definecolor{todocolor}{RGB}{255, 100, 100}
\newcommand{\TODO}[1]{\textcolor{todocolor}{\textbf{[TODO: #1]}}}


% ----- SUMMARY -----
% ============================================================================
% Summary Section - Structure Level Template
% MCM-Analysis Skill v1.3.0
% ============================================================================

\begin{center}
\Large\textbf{Summary}
\end{center}

% ============================================================================
% 【本节目标】约300-450词,严格1页
% 【结构提示】
%   1. 开篇钩子:1-2句描述问题重要性和背景
%   2. 分任务陈述:每个子问题2-4句(方法+结果)
%   3. 总结段:2-3句综合结论和建议
%   4. 关键词:4-6个术语,分号分隔
% ============================================================================

\begin{abstract}

% --- 开篇钩子 (1-2 sentences) ---
% TODO: 描述问题领域的重要性
% 示例: "[领域]面临着[挑战],这对[利益相关者]具有重要影响。"

% --- 问题一 (2-4 sentences) ---
% TODO: For Task 1, we [developed/established] a [Model Name].
% 包含: 方法概述 + 关键发现/数值结果

% --- 问题二 (2-4 sentences) ---
% TODO: For Task 2, we [applied/extended] [approach].
% 包含: 方法概述 + 关键发现/数值结果

% --- 问题三 (2-4 sentences) ---
% TODO: For Task 3, we [analyzed/evaluated] [aspect].
% 包含: 方法概述 + 关键发现/数值结果

% --- 问题四 (如有) (2-4 sentences) ---
% TODO: For Task 4, we [proposed/recommended] [solution].
% 包含: 方法概述 + 关键发现/数值结果

% --- 总结段 (2-3 sentences) ---
% TODO: 综合结论,强调模型优势和实际应用价值

\end{abstract}

\vspace{1em}
\noindent\textbf{Keywords:} keyword1; keyword2; keyword3; keyword4; keyword5


% ----- INTRODUCTION -----
% ============================================================================
% Introduction Section - Structure Level Template
% MCM-Analysis Skill v1.3.0
% ============================================================================

\section{Introduction}

% ============================================================================
% 【本节目标】2-3页,包含背景、问题重述、工作概述
% 【页面分配】
%   - Problem Background: 1-1.5页
%   - Restatement: 0.5-1页  
%   - Our Work: 0.5页
% ============================================================================

\subsection{Problem Background}

% --- 宏观背景 (1 paragraph) ---
% TODO: 介绍问题所在领域的大背景
% 示例: "In recent years, [领域] has attracted significant attention due to..."

% --- 具体问题域 (1-2 paragraphs) ---
% TODO: 聚焦到具体问题,引用相关研究或数据
% 示例: 统计数据、现状描述、存在的挑战

% --- 问题重要性 (1 paragraph) ---
% TODO: 为什么解决这个问题很重要?对谁有影响?

\subsection{Restatement of the Problem}

% TODO: 用自己的话重述问题要求,使用项目符号列表
% 格式:
% For the problem of [领域], we develop mathematical models to:
% \begin{itemize}
%     \item \textbf{Task 1:} [用自己的话重述问题1]
%     \item \textbf{Task 2:} [用自己的话重述问题2]
%     \item \textbf{Task 3:} [用自己的话重述问题3]
%     \item \textbf{Task 4:} [用自己的话重述问题4] (如有)
% \end{itemize}

\subsection{Our Work}

% TODO: 工作流程图(推荐)
% \begin{figure}[H]
%     \centering
%     % \includegraphics[width=0.9\textwidth]{figures/workflow.pdf}
%     \caption{Overview of our modeling approach}
%     \label{fig:workflow}
% \end{figure}

% TODO: 简要说明论文结构
% 示例: "The remainder of this paper is organized as follows: Section 2 presents..."


% ----- ASSUMPTIONS -----
% ============================================================================
% Assumptions Section - Structure Level Template
% MCM-Analysis Skill v1.3.0
% ============================================================================

\section{Assumptions and Justifications}

% ============================================================================
% 【本节目标】1-1.5页,包含假设列表和符号表
% 【假设数量】通常5-8个,每个都需要合理性论证
% ============================================================================

We make the following assumptions to simplify the problem while maintaining practical validity:

\begin{enumerate}[label=\textbf{A\arabic*.}]
    
    % --- 假设1: 数据可靠性 ---
    \item \textbf{Data Reliability:}
    % TODO: 关于数据来源和质量的假设
    
    \textit{Justification:} % TODO: 为什么这个假设合理
    
    % --- 假设2: 系统边界 ---
    \item \textbf{System Boundary:}
    % TODO: 关于系统范围和边界条件的假设
    
    \textit{Justification:} % TODO: 为什么这个假设合理
    
    % --- 假设3: 行为假设 ---
    \item \textbf{Behavioral Assumption:}
    % TODO: 关于参与者/系统行为的假设(如理性决策)
    
    \textit{Justification:} % TODO: 为什么这个假设合理
    
    % --- 假设4: 简化假设 ---
    \item \textbf{Simplification:}
    % TODO: 为简化模型而做的假设
    
    \textit{Justification:} % TODO: 为什么这个简化不影响结论
    
    % --- 假设5: 时间/空间假设 ---
    \item \textbf{Temporal/Spatial Scope:}
    % TODO: 关于时间范围或空间范围的假设
    
    \textit{Justification:} % TODO: 为什么这个假设合理

\end{enumerate}

\subsection{Notation}

% TODO: 填充符号表,包含所有重要变量
\begin{table}[H]
\centering
\begin{tabular}{clc}
\toprule
\textbf{Symbol} & \textbf{Description} & \textbf{Unit} \\
\midrule
$x$ & [Description of variable x] & [unit] \\
$y$ & [Description of variable y] & [unit] \\
$\alpha$ & [Description of parameter alpha] & - \\
$\beta$ & [Description of parameter beta] & - \\
% TODO: 添加更多符号
\bottomrule
\end{tabular}
\caption{Notation used in this paper}
\label{tab:notation}
\end{table}

\textit{Note: Additional variables will be defined in context within each section.}


% ----- MODEL -----
% ============================================================================
% Model Development Section - Structure Level Template
% MCM-Analysis Skill v1.3.0
% ============================================================================

\section{Model Development}

% ============================================================================
% 【本节目标】8-12页,论文核心部分
% 【结构建议】
%   - 模型概述: 0.5-1页(含框架图)
%   - 每个子模型: 2-4页(数学公式 + 求解方法)
% ============================================================================

\subsection{Model Overview}

% TODO: 模型框架图(强烈推荐)
% \begin{figure}[H]
%     \centering
%     % \includegraphics[width=0.85\textwidth]{figures/model_framework.pdf}
%     \caption{Overall framework of our modeling approach}
%     \label{fig:framework}
% \end{figure}

% TODO: 简述模型的整体思路和各部分之间的关系

% ============================================================================
% 模型1 - 对应问题1
% ============================================================================
\subsection{Model I: [Model Name]}
% TODO: 为模型起一个描述性的名字,如 "Discrete Population Dynamics Model"

\subsubsection{Model Formulation}

% TODO: 核心数学公式
% 示例:
% \begin{equation}
%     \frac{dN}{dt} = rN\left(1 - \frac{N}{K}\right)
%     \label{eq:model1}
% \end{equation}

% TODO: 解释公式中每个变量的含义

\subsubsection{Solution Method}

% TODO: 描述如何求解模型
% 选项: 解析解、数值方法、优化算法、仿真等

% TODO: 如需要,添加算法伪代码
% \begin{algorithm}[H]
% \caption{Algorithm Name}
% \begin{algorithmic}[1]
%     \State Initialize parameters
%     \While{not converged}
%         \State Update step
%     \EndWhile
%     \State \Return solution
% \end{algorithmic}
% \end{algorithm}

% ============================================================================
% 模型2 - 对应问题2(如需要)
% ============================================================================
\subsection{Model II: [Model Name]}

\subsubsection{Model Formulation}
% TODO: 第二个模型的数学公式

\subsubsection{Solution Method}
% TODO: 求解方法

% ============================================================================
% 模型3 - 对应问题3(如需要)
% ============================================================================
\subsection{Model III: [Model Name]}

\subsubsection{Model Formulation}
% TODO: 第三个模型的数学公式

\subsubsection{Solution Method}
% TODO: 求解方法

% ============================================================================
% 可选: 模型整合/扩展
% ============================================================================
% \subsection{Model Extension}
% TODO: 如果问题要求扩展或综合模型


% ----- RESULTS -----
% ============================================================================
% Results Section - Structure Level Template
% MCM-Analysis Skill v1.3.0
% ============================================================================

\section{Results and Analysis}

% ============================================================================
% 【本节目标】4-5页,展示和解释结果
% 【结构建议】
%   - 数据描述: 0.5页
%   - 每个问题的结果: 1-1.5页
%   - 综合讨论: 0.5-1页
% ============================================================================

\subsection{Data Description}

% TODO: 描述使用的数据来源和预处理
% 包含: 数据来源、规模、时间范围、预处理步骤

% TODO: 数据统计表(推荐)
% \begin{table}[H]
% \centering
% \begin{tabular}{lcc}
% \toprule
% \textbf{Variable} & \textbf{Range} & \textbf{Description} \\
% \midrule
% ... & ... & ... \\
% \bottomrule
% \end{tabular}
% \caption{Summary statistics of the dataset}
% \label{tab:data}
% \end{table}

\subsection{Results for Task 1}

% TODO: 问题1的主要结果

% --- 结果可视化 ---
% \begin{figure}[H]
%     \centering
%     % \includegraphics[width=0.8\textwidth]{figures/result1.pdf}
%     \caption{[Descriptive caption for Task 1 results]}
%     \label{fig:result1}
% \end{figure}

% --- 数值结果表 ---
% \begin{table}[H]
% \centering
% \begin{tabular}{ccc}
% \toprule
% \textbf{Metric} & \textbf{Value} & \textbf{Unit} \\
% \midrule
% ... & ... & ... \\
% \bottomrule
% \end{tabular}
% \caption{Quantitative results for Task 1}
% \label{tab:result1}
% \end{table}

% TODO: 结果解释 - 这些数字意味着什么?

\subsection{Results for Task 2}

% TODO: 问题2的主要结果(结构同上)

% --- 结果可视化 ---
% \begin{figure}[H]
%     \centering
%     % \includegraphics[width=0.8\textwidth]{figures/result2.pdf}
%     \caption{[Descriptive caption for Task 2 results]}
%     \label{fig:result2}
% \end{figure}

% TODO: 结果解释

\subsection{Results for Task 3}

% TODO: 问题3的主要结果(结构同上)

% --- 结果可视化 ---
% \begin{figure}[H]
%     \centering
%     % \includegraphics[width=0.8\textwidth]{figures/result3.pdf}
%     \caption{[Descriptive caption for Task 3 results]}
%     \label{fig:result3}
% \end{figure}

% TODO: 结果解释

\subsection{Discussion}

% TODO: 综合讨论
% 内容建议:
% - 结果是否符合预期?
% - 各问题结果之间的联系
% - 与现有研究/数据的比较
% - 实际应用意义


% ----- SENSITIVITY -----
% ============================================================================
% Sensitivity Analysis Section - Structure Level Template
% MCM-Analysis Skill v1.3.0
% ============================================================================

\section{Sensitivity Analysis}

% ============================================================================
% 【本节目标】1-1.5页,验证模型鲁棒性
% 【关键内容】
%   - 参数敏感性测试 (±5%, ±10%)
%   - 模型验证
%   - 鲁棒性结论
% ============================================================================

\subsection{Parameter Sensitivity}

% TODO: 选择2-4个关键参数进行敏感性分析

% --- 敏感性分析结果图 ---
% \begin{figure}[H]
%     \centering
%     % \includegraphics[width=0.8\textwidth]{figures/sensitivity.pdf}
%     \caption{Sensitivity analysis of key parameters}
%     \label{fig:sensitivity}
% \end{figure}

% TODO: 敏感性分析结果表
% \begin{table}[H]
% \centering
% \begin{tabular}{lccc}
% \toprule
% \textbf{Parameter} & \textbf{-10\%} & \textbf{Baseline} & \textbf{+10\%} \\
% \midrule
% Parameter 1 & [result] & [result] & [result] \\
% Parameter 2 & [result] & [result] & [result] \\
% \bottomrule
% \end{tabular}
% \caption{Impact of parameter perturbation on model outputs}
% \label{tab:sensitivity}
% \end{table}

% TODO: 解释哪些参数影响大,哪些影响小

\subsection{Model Validation}

% TODO: 模型验证方法
% 选项:
% - 与历史数据对比
% - 交叉验证
% - 极端情况测试
% - 合理性检查

% --- 验证结果 ---
% \begin{figure}[H]
%     \centering
%     % \includegraphics[width=0.7\textwidth]{figures/validation.pdf}
%     \caption{Comparison between model predictions and actual data}
%     \label{fig:validation}
% \end{figure}

\subsection{Robustness Conclusion}

% TODO: 总结模型鲁棒性
% 示例: "Our sensitivity analysis demonstrates that the model is robust 
% to reasonable variations in key parameters. The output changes by less 
% than X% when parameters vary within ±10%."


% ----- STRENGTHS -----
% ============================================================================
% Strengths and Weaknesses Section - Structure Level Template
% MCM-Analysis Skill v1.3.0
% ============================================================================

\section{Strengths and Weaknesses}

% ============================================================================
% 【本节目标】0.5-1页,客观评估模型
% 【平衡原则】优点和缺点数量大致相当,缺点要提供改进方向
% ============================================================================

\subsection{Strengths}

\begin{itemize}
    % TODO: 列出3-5个优点,每个都要有具体解释
    
    \item \textbf{[Strength 1 Title]:}
    % TODO: 具体说明为什么这是优点
    
    \item \textbf{[Strength 2 Title]:}
    % TODO: 具体说明为什么这是优点
    
    \item \textbf{[Strength 3 Title]:}
    % TODO: 具体说明为什么这是优点
    
    % 常见优点类型:
    % - Comprehensive consideration of multiple factors
    % - Supported by real-world data
    % - Robust to parameter variations
    % - Practical applicability
    % - Novel approach / methodology innovation
    % - Validated against known benchmarks
    
\end{itemize}

\subsection{Weaknesses}

\begin{itemize}
    % TODO: 列出2-4个缺点,每个都要提供可能的改进方向
    
    \item \textbf{[Weakness 1 Title]:}
    % TODO: 描述局限性
    
    \textit{Potential Improvement:} % TODO: 如何改进
    
    \item \textbf{[Weakness 2 Title]:}
    % TODO: 描述局限性
    
    \textit{Potential Improvement:} % TODO: 如何改进
    
    \item \textbf{[Weakness 3 Title]:}
    % TODO: 描述局限性
    
    \textit{Potential Improvement:} % TODO: 如何改进
    
    % 常见缺点类型:
    % - Simplified assumptions may not capture all real-world complexity
    % - Limited by available data quality/quantity
    % - Computational complexity restricts scalability
    % - Requires domain expertise for parameter tuning
    % - Geographic/temporal scope limitations
    
\end{itemize}


% ----- CONCLUSION -----
% ============================================================================
% Conclusion Section - Structure Level Template
% MCM-Analysis Skill v1.3.0
% ============================================================================

\section{Conclusions}

% ============================================================================
% 【本节目标】1-2页,总结和建议
% 【结构】
%   - 主要发现 (0.5页)
%   - 建议 (0.5页)
%   - 未来工作 (0.25页)
% ============================================================================

\subsection{Summary of Findings}

% TODO: 总结3-5个关键发现,直接回应问题要求

% 格式建议:
% \begin{enumerate}
%     \item \textbf{Finding 1:} [具体发现,包含数值]
%     \item \textbf{Finding 2:} [具体发现,包含数值]
%     \item \textbf{Finding 3:} [具体发现,包含数值]
% \end{enumerate}

\subsection{Recommendations}

% TODO: 基于分析结果提出可操作的建议

% 格式建议(根据问题要求可能需要写成备忘录格式):
% Based on our analysis, we recommend the following:
% \begin{enumerate}
%     \item [具体、可操作的建议1]
%     \item [具体、可操作的建议2]
%     \item [具体、可操作的建议3]
% \end{enumerate}

\subsection{Future Work}

% TODO: 简述未来可以扩展的方向

% 内容建议:
% - 如何获取更多/更好的数据
% - 模型可以如何改进
% - 其他可以考虑的因素
% - 实际应用中的注意事项



\end{document}
