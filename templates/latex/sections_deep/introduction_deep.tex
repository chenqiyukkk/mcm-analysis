% ============================================================
% MCM-Analysis v2.1 Deep Template: Introduction
% 基于2024 O奖论文结构设计
% ============================================================

\section{Introduction}

% ============================================================
% 1.1 Problem Background
% 【字数目标: 300-500字】
% 【必须包含】:
%   - 领域宏观背景 (1段)
%   - 具体问题的重要性 (1-2段)
%   - 真实世界的挑战 (1段)
% 【建议】: 包含一张真实图片 (Figure 1)
% 【引用要求】: 至少1个权威数据源
% ============================================================
\subsection{Problem Background}

% [LLM生成指导]
% 开头模式: "In [domain], [phenomenon/problem] has long been..."
% 或: "[Topic] plays a crucial role in [broader context]..."
% 
% 示例 (2024 O奖):
% "In the world, the lamprey can be regarded as a 'star creature' 
%  that has attracted much attention. On one hand... on the other hand..."

% TODO: 插入背景图片
% \begin{figure}[htbp]
%     \centering
%     \includegraphics[width=0.6\textwidth]{figures/background.png}
%     \caption{[描述性标题]}
%     \label{fig:background}
% \end{figure}

[在此生成Problem Background内容]

% ============================================================
% 1.2 Literature Review (O奖必需!)
% 【字数目标: 200-300字】
% 【必须包含】:
%   - 前人研究概述 (1段)
%   - 现有方法的优缺点对比 (表格或图)
%   - 研究空白/我们的创新点
% 【引用要求】: 至少3篇相关文献
% 【建议】: 使用对比表格 (Figure 2)
% ============================================================
\subsection{Literature Review}

% [LLM生成指导]
% 结构:
% 1. "In recent years, research on [topic] has focused on..."
% 2. "Several approaches have been proposed: [Method A], [Method B]..."
% 3. "However, [limitations of existing methods]..."
% 4. 对比表格 (见下方模板)

% TODO: 文献对比图
% \begin{figure}[htbp]
%     \centering
%     \includegraphics[width=0.8\textwidth]{figures/literature_comparison.png}
%     \caption{Comparison of existing approaches}
%     \label{fig:literature}
% \end{figure}

% 或使用表格:
% \begin{table}[htbp]
%     \centering
%     \caption{Comparison of Related Methods}
%     \begin{tabular}{lccc}
%         \toprule
%         Method & Advantage & Disadvantage & Applicability \\
%         \midrule
%         Method A & ... & ... & ... \\
%         Method B & ... & ... & ... \\
%         Our Approach & ... & ... & ... \\
%         \bottomrule
%     \end{tabular}
%     \label{tab:literature}
% \end{table}

[在此生成Literature Review内容]

% ============================================================
% 1.3 Restatement of the Problem
% 【字数目标: 150-250字】
% 【必须包含】:
%   - 用自己的话重述问题
%   - 每个Task用bullet point列出
% 【格式】: 使用enumerate或itemize
% ============================================================
\subsection{Restatement of the Problem}

% [LLM生成指导]
% 开头: "Under the above background, in order to [goal], 
%        we will study [topic] and accomplish the following tasks:"
% 
% 然后用enumerate列出每个Task

Under the above background, in order to explore [主要研究目标], our group will study [研究主题] and accomplish the following tasks:

\begin{enumerate}
    \item \textbf{Task 1:} [用自己的话重述Task 1]
    \item \textbf{Task 2:} [用自己的话重述Task 2]
    \item \textbf{Task 3:} [用自己的话重述Task 3]
    % 根据实际Task数量增减
\end{enumerate}

% ============================================================
% 1.4 Our Work (O奖核心!)
% 【字数目标: 150-300字】
% 【必须包含】:
%   - 工作流程思维导图 (Figure 3) ⭐关键
%   - 简要说明整体方法
%   - 论文结构概述
% 【图表要求】: 必须有Figure!
% ============================================================
\subsection{Our Work}

% [LLM生成指导]
% 1. 简述整体方法
% 2. 引用工作流程图
% 3. 简述论文结构

Our solution framework is illustrated in Figure \ref{fig:workflow}.

\begin{figure}[htbp]
    \centering
    \includegraphics[width=0.9\textwidth]{figures/workflow.png}
    \caption{Overview of our modeling approach}
    \label{fig:workflow}
\end{figure}

% 简述各模型:
% "We establish [Model I Name] to address Task 1, which...
%  For Task 2, we develop [Model II Name] that...
%  Finally, we perform sensitivity analysis to validate..."

[在此生成Our Work内容描述]

% ============================================================
% 引用占位 (LLM填充时替换)
% ============================================================
% 在正文中使用: \cite{ref1}, \cite{ref2} 等
% 确保在references.bib中有对应条目
