% ============================================================
% MCM-Analysis v2.1 Deep Template: Model Development
% 基于2024 O奖论文结构设计
% ============================================================

\section{Model Preparation}

% ============================================================
% 2.1 Assumptions and Justifications
% 【数量目标: 4-6个假设】
% 【格式】: 每个假设 = 陈述 + 理由
% ============================================================
\subsection{Assumptions and Justifications}

% [LLM生成指导]
% 格式:
% • Assumption N: [清晰陈述假设内容]
%   Justification: [为什么这个假设合理 - 1-3句话]

\begin{itemize}
    \item \textbf{Assumption 1: [假设名称]}
    
    [假设内容陈述]
    
    \textit{Justification:} [理由说明,可引用文献支持]
    
    \item \textbf{Assumption 2: [假设名称]}
    
    [假设内容陈述]
    
    \textit{Justification:} [理由说明]
    
    \item \textbf{Assumption 3: [假设名称]}
    
    [假设内容陈述]
    
    \textit{Justification:} [理由说明]
    
    % 继续添加更多假设...
\end{itemize}

% ============================================================
% 2.2 Notations
% 【格式】: 标准三列表格
% ============================================================
\subsection{Notations}

This section introduces the symbols and their descriptions used in this paper.

\begin{table}[H]
\centering
\caption{Notations Table}
\begin{tabular}{clc}
\toprule
\textbf{Symbol} & \textbf{Definition} & \textbf{Unit} \\
\midrule
$x$ & [变量x的定义] & [单位] \\
$y$ & [变量y的定义] & [单位] \\
$\alpha$ & [参数alpha的定义] & - \\
$\beta$ & [参数beta的定义] & - \\
% 继续添加更多符号...
\bottomrule
\end{tabular}
\label{tab:notations}
\end{table}

\textit{Note: Some variables not listed above will be defined in detail in each section.}

% ============================================================
% 2.3 Data Collection and Visualization (如果有数据)
% 【可选】: 仅当问题提供数据时使用
% ============================================================
\subsection{Data Collection and Visualization}

% [LLM生成指导]
% 如果问题提供了数据文件,需要:
% 1. 列出数据来源表
% 2. 展示数据分布可视化

% 数据来源表模板:
% \begin{table}[htbp]
%     \centering
%     \caption{Data Sources}
%     \begin{tabular}{ll}
%         \toprule
%         \textbf{Data Type} & \textbf{Source} \\
%         \midrule
%         [数据类型1] & [来源/URL] \\
%         [数据类型2] & [来源/URL] \\
%         \bottomrule
%     \end{tabular}
%     \label{tab:data_sources}
% \end{table}

% 数据分布图:
% \begin{figure}[htbp]
%     \centering
%     \includegraphics[width=0.7\textwidth]{figures/data_distribution.png}
%     \caption{Distribution of [数据名称]}
%     \label{fig:data_dist}
% \end{figure}

[如果有数据,在此生成数据描述和可视化]

% ============================================================
\section{Model Establishment}
% ============================================================

% ============================================================
% 3.1 Model I: [模型名称]
% 【结构】: 每个模型独立subsection
% 【必须包含】: 模型思维导图
% ============================================================
\subsection{Model I: [模型名称]}

% [LLM生成指导]
% 每个主模型应该有:
% 1. 模型思维导图 (Figure)
% 2. 多个sub-subsection解释各部分
% 3. 核心公式 + 推导过程
% 4. 参数说明

% 模型思维导图
\begin{figure}[htbp]
    \centering
    \includegraphics[width=0.85\textwidth]{figures/model1_mindmap.png}
    \caption{Framework of Model I: [模型名称]}
    \label{fig:model1_mindmap}
\end{figure}

% ---- 子模型/组件 ----
\subsubsection{[组件1名称]}

% [物理/数学原理说明]
% 如果有物理过程,添加示意图:
% \begin{figure}[htbp]
%     \centering
%     \includegraphics[width=0.6\textwidth]{figures/mechanism.png}
%     \caption{Schematic of [机制名称]}
%     \label{fig:mechanism}
% \end{figure}

[组件1的详细说明]

% 核心公式
The [物理量] can be expressed as:
\begin{equation}
    [公式内容]
    \label{eq:model1_eq1}
\end{equation}
where $[变量1]$ represents [含义], and $[变量2]$ denotes [含义].

\subsubsection{[组件2名称]}

[组件2的详细说明]

\begin{equation}
    [公式内容]
    \label{eq:model1_eq2}
\end{equation}

% ---- 继续添加更多子模型 ----

% ============================================================
% 3.2 Model II: [模型名称] (如果有第二个模型)
% ============================================================
\subsection{Model II: [模型名称]}

% 同样的结构:
% 1. 模型思维导图
% 2. 子组件说明
% 3. 核心公式

\begin{figure}[htbp]
    \centering
    \includegraphics[width=0.85\textwidth]{figures/model2_mindmap.png}
    \caption{Framework of Model II: [模型名称]}
    \label{fig:model2_mindmap}
\end{figure}

\subsubsection{[组件名称]}

[组件说明和公式]

% ============================================================
% 3.N Model Overview / Integration
% 【可选】: 如果有多个模型,说明如何整合
% ============================================================
\subsection{Model Integration}

% [LLM生成指导]
% 说明各模型之间的关系
% 可以用流程图展示数据流

[说明各模型如何协同工作]

% ============================================================
% 算法伪代码 (O奖常见!)
% 【建议】: 使用algorithm环境
% ============================================================
% 需要在preamble中添加: \usepackage{algorithm, algorithmic}

% \begin{algorithm}[htbp]
%     \caption{[算法名称]}
%     \label{alg:main}
%     \begin{algorithmic}[1]
%         \REQUIRE [输入参数]
%         \ENSURE [输出结果]
%         \STATE Initialize parameters
%         \FOR{$t = 1$ to $T$}
%             \STATE [步骤1]
%             \STATE [步骤2]
%             \IF{[条件]}
%                 \STATE [操作]
%             \ENDIF
%         \ENDFOR
%         \RETURN [返回值]
%     \end{algorithmic}
% \end{algorithm}
