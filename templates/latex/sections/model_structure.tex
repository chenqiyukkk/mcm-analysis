% ============================================================================
% Model Development Section - Structure Level Template
% MCM-Analysis Skill v1.3.0
% ============================================================================

\section{Model Development}

% ============================================================================
% 【本节目标】8-12页,论文核心部分
% 【结构建议】
%   - 模型概述: 0.5-1页(含框架图)
%   - 每个子模型: 2-4页(数学公式 + 求解方法)
% ============================================================================

\subsection{Model Overview}

% TODO: 模型框架图(强烈推荐)
% \begin{figure}[H]
%     \centering
%     % \includegraphics[width=0.85\textwidth]{figures/model_framework.pdf}
%     \caption{Overall framework of our modeling approach}
%     \label{fig:framework}
% \end{figure}

% TODO: 简述模型的整体思路和各部分之间的关系

% ============================================================================
% 模型1 - 对应问题1
% ============================================================================
\subsection{Model I: [Model Name]}
% TODO: 为模型起一个描述性的名字,如 "Discrete Population Dynamics Model"

\subsubsection{Model Formulation}

% TODO: 核心数学公式
% 示例:
% \begin{equation}
%     \frac{dN}{dt} = rN\left(1 - \frac{N}{K}\right)
%     \label{eq:model1}
% \end{equation}

% TODO: 解释公式中每个变量的含义

\subsubsection{Solution Method}

% TODO: 描述如何求解模型
% 选项: 解析解、数值方法、优化算法、仿真等

% TODO: 如需要,添加算法伪代码
% \begin{algorithm}[H]
% \caption{Algorithm Name}
% \begin{algorithmic}[1]
%     \State Initialize parameters
%     \While{not converged}
%         \State Update step
%     \EndWhile
%     \State \Return solution
% \end{algorithmic}
% \end{algorithm}

% ============================================================================
% 模型2 - 对应问题2(如需要)
% ============================================================================
\subsection{Model II: [Model Name]}

\subsubsection{Model Formulation}
% TODO: 第二个模型的数学公式

\subsubsection{Solution Method}
% TODO: 求解方法

% ============================================================================
% 模型3 - 对应问题3(如需要)
% ============================================================================
\subsection{Model III: [Model Name]}

\subsubsection{Model Formulation}
% TODO: 第三个模型的数学公式

\subsubsection{Solution Method}
% TODO: 求解方法

% ============================================================================
% 可选: 模型整合/扩展
% ============================================================================
% \subsection{Model Extension}
% TODO: 如果问题要求扩展或综合模型
