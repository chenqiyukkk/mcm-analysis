% ============================================================================
% Sensitivity Analysis Section - Draft Level Template (Chinese)
% MCM-Analysis Skill v1.3.0
% ============================================================================

\section{Sensitivity Analysis}

% ============================================================================
% 【以下为中文初稿模板,请根据具体问题修改后翻译为英文】
% ============================================================================

为验证模型的鲁棒性,我们对关键参数进行了灵敏度分析。

\subsection{Parameter Sensitivity}

我们选取了以下[数量]个关键参数进行灵敏度测试:
[参数1名称]、[参数2名称]和[参数3名称]。
对每个参数分别进行$\pm 5\%$和$\pm 10\%$的扰动,
观察模型输出的变化情况。

表~\ref{tab:sensitivity}~展示了灵敏度分析的结果。

\begin{table}[H]
\centering
\begin{tabular}{lccccc}
\toprule
\textbf{参数} & \textbf{-10\%} & \textbf{-5\%} & \textbf{基准值} & \textbf{+5\%} & \textbf{+10\%} \\
\midrule
参数1 & [值] & [值] & [值] & [值] & [值] \\
参数2 & [值] & [值] & [值] & [值] & [值] \\
参数3 & [值] & [值] & [值] & [值] & [值] \\
\bottomrule
\end{tabular}
\caption{参数灵敏度分析结果}
\label{tab:sensitivity}
\end{table}

从表中可以看出:
\begin{itemize}
    \item [参数1]的变化对结果影响最大,当参数增加10\%时,输出变化[百分比]。
    \item [参数2]的影响相对较小,输出变化控制在[百分比]以内。
    \item [参数3]对模型输出几乎没有影响,说明模型对该参数不敏感。
\end{itemize}

图~\ref{fig:sensitivity}~直观展示了参数敏感性的对比。

% \begin{figure}[H]
%     \centering
%     % \includegraphics[width=0.8\textwidth]{figures/sensitivity_tornado.pdf}
%     \caption{参数灵敏度龙卷风图}
%     \label{fig:sensitivity}
% \end{figure}

\subsection{Model Validation}

为进一步验证模型的有效性,我们采用了以下验证方法:

\textbf{方法一:历史数据对比}

我们将模型预测结果与[时间段]的历史数据进行对比。
如图~\ref{fig:validation}~所示,模型预测值与实际值的相关系数达到[值],
平均绝对误差(MAE)为[值],说明模型具有良好的预测精度。

% \begin{figure}[H]
%     \centering
%     % \includegraphics[width=0.7\textwidth]{figures/validation.pdf}
%     \caption{模型验证:预测值与实际值对比}
%     \label{fig:validation}
% \end{figure}

\textbf{方法二:极端情况测试}

我们测试了[极端情况1]和[极端情况2]下模型的表现。
结果表明,在这些边界条件下,模型仍能给出合理的结果,
没有出现数值溢出或不收敛的情况。

\subsection{Robustness Conclusion}

综合灵敏度分析和模型验证的结果,我们可以得出以下结论:

\begin{enumerate}
    \item 模型对大多数参数的变化具有较好的鲁棒性,
          输出变化幅度在可接受范围内。
    \item [参数1]是模型中最敏感的参数,在实际应用中应重点关注其取值的准确性。
    \item 模型预测结果与历史数据吻合良好,验证了模型的有效性。
    \item 在极端条件下,模型仍能保持稳定,具有较强的适应性。
\end{enumerate}
