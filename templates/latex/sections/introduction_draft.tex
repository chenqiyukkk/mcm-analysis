% ============================================================================
% Introduction Section - Draft Level Template (Chinese)
% MCM-Analysis Skill v1.3.0
% ============================================================================

\section{Introduction}

% ============================================================================
% 【以下为中文初稿模板,请根据具体问题修改后翻译为英文】
% ============================================================================

\subsection{Problem Background}

% --- 宏观背景 ---
随着[领域/技术]的快速发展,[问题主题]已成为[学术界/工业界/政府]关注的焦点问题。
[引用统计数据或权威来源]显示,[具体现状描述],这一趋势在[时间范围]内持续加剧。

% --- 问题具体化 ---
具体而言,[问题核心矛盾]导致了[负面影响]。
例如,[具体案例或数据]表明[问题严重性]。
现有研究主要集中在[研究方向1]和[研究方向2],但在[研究空白]方面仍存在不足。

% --- 问题重要性 ---
解决这一问题具有重要的[理论/实践/经济/社会]意义。
一方面,[意义1];另一方面,[意义2]。
因此,建立科学有效的数学模型对[目标]具有重要价值。

\subsection{Restatement of the Problem}

基于以上背景,我们针对[问题领域]建立数学模型,解决以下关键问题:

\begin{itemize}
    \item \textbf{问题一:}[用自己的语言重述问题一的核心要求]
    
    \item \textbf{问题二:}[用自己的语言重述问题二的核心要求]
    
    \item \textbf{问题三:}[用自己的语言重述问题三的核心要求]
    
    \item \textbf{问题四:}[用自己的语言重述问题四的核心要求](如有)
\end{itemize}

\subsection{Our Work}

针对上述问题,我们的工作可以概括为以下几个方面:

首先,我们对问题进行了深入分析,明确了[核心挑战]和[关键约束]。
在此基础上,我们建立了合理的假设体系,为后续建模奠定基础。

其次,我们构建了[模型数量]个数学模型:
\begin{enumerate}
    \item \textbf{[模型一名称]:}用于解决问题一,核心思想是[一句话概括]。
    \item \textbf{[模型二名称]:}用于解决问题二,主要方法是[一句话概括]。
    \item \textbf{[模型三名称]:}用于解决问题三,关键技术是[一句话概括]。
\end{enumerate}

最后,我们通过灵敏度分析验证了模型的鲁棒性,并提出了针对性的建议。

% --- 工作流程图 ---
% \begin{figure}[H]
%     \centering
%     % \includegraphics[width=0.9\textwidth]{figures/workflow.pdf}
%     \caption{本文的整体研究框架}
%     \label{fig:workflow}
% \end{figure}

本文的结构安排如下:
第2节介绍模型假设和符号定义;
第3节详细阐述模型的构建过程;
第4节展示模型结果并进行分析;
第5节进行灵敏度分析;
第6节总结优缺点;
第7节给出结论和建议。
