% ============================================================================
% Model Development Section - Draft Level Template (Chinese)
% MCM-Analysis Skill v1.3.0
% ============================================================================

\section{Model Development}

% ============================================================================
% 【以下为中文初稿模板,请根据具体问题修改后翻译为英文】
% ============================================================================

\subsection{Model Overview}

本节将详细介绍我们构建的数学模型。
针对题目中的[问题数量]个问题,我们分别建立了相应的模型进行分析。
图~\ref{fig:framework}~展示了整体建模框架。

% \begin{figure}[H]
%     \centering
%     % \includegraphics[width=0.85\textwidth]{figures/model_framework.pdf}
%     \caption{整体建模框架}
%     \label{fig:framework}
% \end{figure}

我们的建模思路如下:
首先,通过[数据分析/理论推导]确定问题的关键要素;
其次,基于[理论基础]构建数学模型;
最后,采用[求解方法]获得最优解或数值结果。

% ============================================================================
% 模型一
% ============================================================================
\subsection{Model I: [模型一名称]}

\subsubsection{问题分析}

问题一要求我们[问题一的核心目标]。
通过分析,我们发现该问题的关键在于[核心挑战]。
因此,我们选择采用[方法名称]来解决这一问题。

\subsubsection{模型构建}

基于上述分析,我们建立如下数学模型。

设[变量定义],则目标函数可以表示为:

\begin{equation}
    % 目标函数示例
    \min_{x} \quad f(x) = \sum_{i=1}^{n} c_i x_i
    \label{eq:objective1}
\end{equation}

约束条件为:

\begin{equation}
    \begin{aligned}
        & \text{s.t.} \quad \sum_{i=1}^{n} a_{ij} x_i \leq b_j, \quad j = 1, 2, \ldots, m \\
        & \qquad\quad x_i \geq 0, \quad i = 1, 2, \ldots, n
    \end{aligned}
    \label{eq:constraint1}
\end{equation}

其中,$c_i$表示[含义],$a_{ij}$表示[含义],$b_j$表示[含义]。

\subsubsection{求解方法}

对于模型~(\ref{eq:objective1})-(\ref{eq:constraint1}),我们采用[算法名称]进行求解。
该算法的基本步骤如下:

\begin{enumerate}
    \item 初始化:设置初始解$x^{(0)}$和参数$\theta$;
    \item 迭代:按照[更新规则]更新解;
    \item 判断:若满足收敛条件则停止,否则返回步骤2;
    \item 输出:返回最优解$x^*$。
\end{enumerate}

% ============================================================================
% 模型二
% ============================================================================
\subsection{Model II: [模型二名称]}

\subsubsection{问题分析}

问题二在问题一的基础上,进一步要求[问题二的核心目标]。
这需要我们考虑[额外因素]的影响。

\subsubsection{模型构建}

在模型一的基础上,我们引入[新变量/新约束],构建扩展模型:

\begin{equation}
    % 模型二公式示例
    \frac{dN}{dt} = rN\left(1 - \frac{N}{K}\right) - hN
    \label{eq:model2}
\end{equation}

其中,$r$为[含义],$K$为[含义],$h$为[含义]。

\subsubsection{求解方法}

对于微分方程~(\ref{eq:model2}),我们采用[数值方法]进行求解,
时间步长设为$\Delta t = [值]$。

% ============================================================================
% 模型三
% ============================================================================
\subsection{Model III: [模型三名称]}

\subsubsection{问题分析}

问题三要求我们[问题三的核心目标]。
该问题的特点是[问题特性],适合采用[方法类别]进行分析。

\subsubsection{模型构建}

我们建立以下模型:

\begin{equation}
    % 模型三公式示例
    P(Y = 1 | X) = \frac{1}{1 + e^{-(\beta_0 + \beta_1 X_1 + \ldots + \beta_p X_p)}}
    \label{eq:model3}
\end{equation}

\subsubsection{求解方法}

使用[最大似然估计/梯度下降等方法]估计模型参数,
通过[交叉验证/AIC/BIC]选择最优模型。
