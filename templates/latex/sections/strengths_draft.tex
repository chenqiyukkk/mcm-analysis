% ============================================================================
% Strengths and Weaknesses Section - Draft Level Template (Chinese)
% MCM-Analysis Skill v1.3.0
% ============================================================================

\section{Strengths and Weaknesses}

% ============================================================================
% 【以下为中文初稿模板,请根据具体问题修改后翻译为英文】
% ============================================================================

\subsection{Strengths}

本文所建立的模型具有以下优点:

\begin{itemize}
    \item \textbf{综合性强:}
    我们的模型综合考虑了[因素1]、[因素2]和[因素3]等多个关键因素,
    能够全面刻画[研究对象]的复杂特性。
    相比于仅考虑单一因素的简单模型,我们的方法更接近实际情况。
    
    \item \textbf{科学依据充分:}
    模型的构建基于[理论基础/文献来源],
    参数的选取参考了[权威数据来源],
    确保了模型的科学性和可靠性。
    
    \item \textbf{鲁棒性好:}
    灵敏度分析表明,当关键参数在合理范围内变化时,
    模型输出保持相对稳定。
    这说明模型具有较强的抗干扰能力,适用于不同的实际场景。
    
    \item \textbf{实用性强:}
    我们的模型不仅能够解释现象,还能提供可操作的决策建议。
    [具体应用场景]可以直接参考我们的分析结果进行决策。
    
    \item \textbf{可扩展性:}
    模型框架具有良好的可扩展性。
    如需考虑更多因素,只需在现有框架基础上添加相应模块即可,
    无需重新设计整体架构。
\end{itemize}

\subsection{Weaknesses}

尽管我们的模型具有上述优点,但仍存在一些局限性:

\begin{itemize}
    \item \textbf{假设简化:}
    为使模型可解,我们做出了[具体假设]等简化假设。
    在某些极端情况下,这些假设可能不成立,影响模型的准确性。
    
    \textit{改进方向:}
    未来可以放松[假设]的约束,引入更复杂的机制来处理[情况]。
    
    \item \textbf{数据局限:}
    本研究使用的数据存在[数据局限性描述]的问题。
    例如,[具体数据问题]可能导致[潜在影响]。
    
    \textit{改进方向:}
    若能获取[更好的数据来源],可以进一步提高模型的准确性。
    
    \item \textbf{计算复杂度:}
    当[问题规模描述]较大时,模型的求解时间会显著增加。
    这限制了模型在[大规模应用场景]中的直接应用。
    
    \textit{改进方向:}
    可以采用[优化算法/近似方法/并行计算]等技术提高计算效率。
    
    \item \textbf{外部因素:}
    模型未能充分考虑[外部因素]的影响。
    在实际应用中,这些因素可能对结果产生[影响类型]的影响。
    
    \textit{改进方向:}
    后续研究可以引入[方法]来量化这些外部因素的作用。
\end{itemize}
