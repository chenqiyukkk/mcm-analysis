% ============================================================================
% Assumptions Section - Draft Level Template (Chinese)
% MCM-Analysis Skill v1.3.0
% ============================================================================

\section{Assumptions and Justifications}

% ============================================================================
% 【以下为中文初稿模板,请根据具体问题修改后翻译为英文】
% ============================================================================

为了简化问题并使模型具有可操作性,我们做出以下假设:

\begin{enumerate}[label=\textbf{A\arabic*.}]
    
    \item \textbf{数据可靠性假设:}
    我们假设所使用的数据来源真实可靠,能够反映实际情况。
    
    \textit{合理性说明:}
    本文使用的数据主要来源于[数据来源],该数据源具有权威性,
    被广泛应用于[相关领域]的研究中。
    
    \item \textbf{系统封闭性假设:}
    我们假设研究系统在分析时间范围内相对封闭,不受外部突发因素的显著影响。
    
    \textit{合理性说明:}
    尽管现实中可能存在外部干扰,但在[时间范围]内,
    系统的主要变化规律相对稳定,该假设不会显著影响模型的有效性。
    
    \item \textbf{参数稳定性假设:}
    我们假设模型中的关键参数在分析期间保持相对稳定。
    
    \textit{合理性说明:}
    根据历史数据分析,[参数名称]在[时间范围]内的变化幅度在[百分比]以内,
    可以近似视为常数。此外,我们通过灵敏度分析验证了这一假设的影响。
    
    \item \textbf{理性行为假设:}
    我们假设系统中的[参与者/决策者]遵循理性决策原则,
    以[目标]最大化为行为准则。
    
    \textit{合理性说明:}
    这是[经济学/博弈论/优化理论]中的经典假设,
    虽然现实中可能存在非理性行为,但从整体趋势来看,
    理性假设能够较好地描述[主体]的平均行为特征。
    
    \item \textbf{简化假设:}
    为便于模型求解,我们忽略[次要因素]的影响。
    
    \textit{合理性说明:}
    根据预分析,[次要因素]对结果的影响不超过[百分比],
    忽略该因素可以显著降低模型复杂度,同时保持结果的准确性。

\end{enumerate}

\subsection{Notation}

表~\ref{tab:notation}~列出了本文中使用的主要符号及其含义。

\begin{table}[H]
\centering
\begin{tabular}{clc}
\toprule
\textbf{符号} & \textbf{含义} & \textbf{单位} \\
\midrule
$t$ & 时间变量 & [时间单位] \\
$N$ & 样本数量 / 总数 & - \\
$x_i$ & 第$i$个观测值 / 决策变量 & [相应单位] \\
$\alpha$ & 模型参数1 & - \\
$\beta$ & 模型参数2 & - \\
$f(\cdot)$ & 目标函数 & - \\
$C$ & 成本 / 约束常数 & [货币/相应单位] \\
\bottomrule
\end{tabular}
\caption{本文主要符号说明}
\label{tab:notation}
\end{table}

\textit{注:部分局部变量将在各节中详细说明。}
